\documentclass[14pt]{extarticle}
\usepackage{fontspec}
\usepackage{geometry}
\usepackage[english,russian]{babel}
\usepackage{minted}
\usepackage{xcolor}
\usepackage{titlesec}
\usepackage{graphicx}
\usepackage{float}
\usepackage{indentfirst}
\usepackage{setspace}
\usepackage{hyperref}

% TikZ для диаграмм
\usepackage{tikz}
\usetikzlibrary{shapes,arrows,positioning,er}

% Стили для ключей в диаграммах
\newcommand{\pk}[1]{\textbf{\textcolor{blue}{#1}}}
\newcommand{\fkstyle}[1]{\textit{\textcolor{red}{#1}}}

\setmainfont{Times New Roman}
\setmonofont{Consolas}

\geometry{a4paper,left=30mm,right=15mm,top=20mm,bottom=20mm}

\onehalfspacing
\parindent 1.25cm

\definecolor{codebg}{rgb}{0.95,0.95,0.95}

\titleformat{\section}{\normalfont\fontsize{16}{19}\bfseries\centering}{\thesection}{1em}{}
\titleformat{\subsection}{\normalfont\fontsize{14}{17}\bfseries}{\thesubsection}{1em}{}
\titleformat{\subsubsection}{\normalfont\fontsize{14}{17}\itshape}{\thesubsubsection}{1em}{}

\begin{document}

\begin{titlepage}
    \centering
    {\fontsize{14}{16}\selectfont МИНИСТЕРСТВО НАУКИ И ВЫСШЕГО ОБРАЗОВАНИЯ РОССИЙСКОЙ ФЕДЕРАЦИИ}\\
    {\fontsize{14}{16}\selectfont Федеральное государственное бюджетное образовательное учреждение}\\
    {\fontsize{14}{16}\selectfont высшего образования}\\
    {\fontsize{14}{16}\selectfont «ИРКУТСКИЙ ГОСУДАРСТВЕННЫЙ УНИВЕРСИТЕТ»}\\
    \vspace{1cm}

    {\fontsize{14}{16}\selectfont Факультет фундаментальной и прикладной информатики}\\
    {\fontsize{14}{16}\selectfont Направление: 02.03.02 «Фундаментальная информатика и информационные технологии»}\\
    \vspace{2cm}

    {\fontsize{16}{19}\bfseries ОТЧЕТ}\\
    {\fontsize{16}{19}\bfseries по лабораторной работе №1}\\
    {\fontsize{16}{19}\bfseries «Проектирование информационной модели для реляционных баз данных»}\\
    \vspace{1cm}

    {\fontsize{14}{16}\selectfont Вариант 65}\\
    \vspace{2cm}

    \begin{flushright}
        {\fontsize{14}{16}\selectfont Выполнил:}\\
        {\fontsize{14}{16}\selectfont студент группы ФИИТ-XX-XX}\\
        {\fontsize{14}{16}\selectfont Иванов И.И.}\\
        \vspace{0.5cm}
        {\fontsize{14}{16}\selectfont Проверил:}\\
        {\fontsize{14}{16}\selectfont доцент кафедры ФИИТ}\\
        {\fontsize{14}{16}\selectfont Петров П.П.}\\
    \end{flushright}
    \vfill

    {\fontsize{14}{16}\selectfont Иркутск 2024}
\end{titlepage}

\section*{РЕФЕРАТ}

Отчет по лабораторной работе №1: «Проектирование информационной модели для реляционных баз данных».

\textbf{Количество страниц} - 8, \textbf{количество рисунков} - 3, \textbf{количество источников} - 2.

\textbf{Ключевые слова}: БАЗЫ ДАННЫХ, ER-МОДЕЛЬ, ЛОГИЧЕСКАЯ МОДЕЛЬ, ФИЗИЧЕСКАЯ МОДЕЛЬ, POSTGRESQL, НОРМАЛИЗАЦИЯ, СУРОГАТНЫЙ КЛЮЧ.

\textbf{Объект исследования}: процесс проектирования реляционной базы данных для системы учета контактов и взаимодействий (личная CRM).

\textbf{Цель работы}: разработка корректной информационной модели базы данных, соответствующей требованиям реляционной теории и нормальных форм.

\textbf{Методы исследования}: анализ предметной области, проектирование ER-модели, преобразование в логическую и физическую модели, нормализация базы данных.

\textbf{Основные результаты}: разработана трехуровневая модель базы данных, устранены критические ошибки проектирования, созданы диаграммы для визуализации структуры базы данных.

\tableofcontents

\section{ВВЕДЕНИЕ}

Современные информационные системы требуют тщательного проектирования структур хранения данных. Реляционные базы данных остаются одним из наиболее распространенных подходов к организации хранения структурированной информации. Правильное проектирование схемы базы данных является критически важным этапом, определяющим эффективность, надежность и масштабируемость всей информационной системы.

В данной лабораторной работе рассматривается процесс проектирования базы данных для системы учета контактов и взаимодействий (личной CRM). Такой тип систем широко используется для управления персональными и деловыми контактами, планирования встреч и ведения истории взаимодействий.

Цель работы заключается в разработке корректной информационной модели, соответствующей требованиям реляционной теории и нормальных форм. Для достижения этой цели необходимо решить следующие задачи:

\begin{enumerate}
    \item Проанализировать предметную область и требования к системе
    \item Разработать ER-модель базы данных
    \item Преобразовать ER-модель в логическую модель
    \item Спроектировать физическую модель для СУБД PostgreSQL
    \item Провести нормализацию базы данных
    \item Визуализировать результаты с помощью диаграмм
\end{enumerate}

В работе использован итерационный подход к проектированию с привлечением современных инструментов визуализации и анализа.

\section{ПОСТАНОВКА ЗАДАЧИ}

\subsection{Техническое задание}

Разработать информационную модель базы данных для системы «Учет контактов и взаимодействий (личная CRM)» со следующими характеристиками:

\textbf{Сущности системы}:
\begin{itemize}
    \item Контакты (ФИО, место работы, телефон)
    \item Встречи (дата, тема, место)
    \item Заметки (дата, текст заметки по контакту)
\end{itemize}

\textbf{Бизнес-процессы}: регистрация всех встреч и важных событий, связанных с контактами.

\textbf{Ограничения предметной области}:
\begin{enumerate}
    \item С одним контактом можно организовать несколько встреч
    \item Встречи происходят только с одним контактом
    \item Система предполагает использование СУБД PostgreSQL
\end{enumerate}

\textbf{Выходные документы}:
\begin{enumerate}
    \item Список предстоящих встреч на неделю с указанием контактов и тем, отсортированный по дате и времени
    \item История всех встреч и заметок для заданного контакта, отсортированная по дате
\end{enumerate}

\subsection{Исходная модель}

Первоначально была предложена следующая модель базы данных:

\begin{minted}[bgcolor=codebg,fontsize=\small]{sql}
Contact(family_name, work_place, phone) -- первичный ключ: phone
Meeting(contact_phone, meeting_time, topic, place) -- первичный ключ: meeting_time
Note(pk, contact_phone, meeting_time, note) -- вторичный ключ: (contact_phone, meeting_time)
\end{minted}

Типы данных для атрибутов:
\begin{itemize}
    \item \texttt{family\_name::varchar}
    \item \texttt{work\_place::varchar}
    \item \texttt{phone::Decimal(10)}
    \item \texttt{meeting\_time::datetime}
    \item \texttt{topic::text}
    \item \texttt{place::text}
    \item \texttt{pk::integer}
    \item \texttt{contact\_phone::Decimal(10)}
    \item \texttt{note:text}
\end{itemize}

\section{МЕТОДОЛОГИЯ ПРОЕКТИРОВАНИЯ}

Проектирование базы данных выполнялось с использованием трехуровневой архитектуры:

\subsection{Уровень 1: Концептуальное проектирование}

На данном уровне разрабатывается ER-модель (Entity-Relationship model), которая описывает сущности предметной области и отношения между ними без привязки к конкретной СУБД.

\subsection{Уровень 2: Логическое проектирование}

Преобразование ER-модели в реляционную схему с определением таблиц, атрибутов, первичных и внешних ключей.

\subsection{Уровень 3: Физическое проектирование}

Реализация логической модели в конкретной СУБД (PostgreSQL) с определением типов данных, индексов, ограничений.

Для визуализации моделей использовался язык Mermaid, поддерживаемый системой GitHub в файлах README.md.

\section{РЕЗУЛЬТАТЫ И ОБСУЖДЕНИЕ}

\subsection{Анализ исходной модели}

При анализе первоначальной модели были выявлены следующие критические проблемы:

\begin{enumerate}
    \item \textbf{Некорректный выбор первичных ключей}: использование телефонного номера и времени встречи в качестве первичных ключей нарушает принципы реляционного проектирования
    \item \textbf{Нарушение нормальных форм}: наличие транзитивных зависимостей в таблице заметок
    \item \textbf{Некорректные типы данных}: использование числового типа для хранения телефонных номеров
    \item \textbf{Избыточная сложность}: жесткая привязка заметок одновременно к контактам и встречам
\end{enumerate}

\subsection{Корректировка модели}

В процессе итерационного проектирования с привлечением экспертной системы были внесены следующие исправления:

\subsubsection{Введение суррогатных ключей}

Вместо естественных ключей (\texttt{phone}, \texttt{meeting\_time}) введены суррогатные ключи (\texttt{contact\_id}, \texttt{meeting\_id}):

\begin{minted}[bgcolor=codebg,fontsize=\small]{sql}
-- Было: Contact(phone, family_name, work_place)
-- Стало: Contact(contact_id, family_name, work_place, phone)

-- Было: Meeting(contact_phone, meeting_time, topic, place)
-- Стало: Meeting(meeting_id, contact_id, meeting_time, topic, place)
\end{minted}

\subsubsection{Корректировка типов данных}

\begin{minted}[bgcolor=codebg,fontsize=\small]{sql}
-- Было: phone::Decimal(10)
-- Стало: phone VARCHAR(20)

-- Было: meeting_time::datetime
-- Стало: meeting_time TIMESTAMP
\end{minted}

\subsubsection{Упрощение структуры заметок}

Упрощена структура таблицы заметок путем удаления обязательной привязки к встречам:

\begin{minted}[bgcolor=codebg,fontsize=\small]{sql}
-- Было: Note(pk, contact_phone, meeting_time, note)
-- Стало: Note(note_id, contact_id, note_text, created_at)
\end{minted}

\subsection{ER-модель}


\begin{figure}[H]
\centering
\begin{tikzpicture}[
    node distance=2cm,
    auto,
    entity/.style={rectangle, draw=blue!50, fill=blue!20, thick, minimum width=3cm, minimum height=1cm, align=center},
    relationship/.style={diamond, draw=red!50, fill=red!20, thick, minimum size=1cm, align=center},
    arrow/.style={->, >=stealth', shorten >=1pt, thick}
]

% Сущности
\node[entity] (contact) {CONTACT \\ \scriptsize integer contact\_id PK \\ \scriptsize varchar family\_name \\ \scriptsize varchar work\_place \\ \scriptsize varchar phone};
\node[entity, below left=2cm and 1cm of contact] (meeting) {MEETING \\ \scriptsize integer meeting\_id PK \\ \scriptsize integer contact\_id FK \\ \scriptsize timestamp meeting\_time \\ \scriptsize text topic \\ \scriptsize text place};
\node[entity, below right=2cm and 1cm of contact] (note) {NOTE \\ \scriptsize integer note\_id PK \\ \scriptsize integer contact\_id FK \\ \scriptsize text note\_text \\ \scriptsize timestamp created\_at};

% Связи
\draw[arrow] (contact) -- node[pos=0.3, above, sloped] {1} node[pos=0.7, above, sloped] {N} (meeting);
\draw[arrow] (contact) -- node[pos=0.3, above, sloped] {1} node[pos=0.7, above, sloped] {N} (note);

% Подписи отношений
\node[above=0.5cm of meeting] {\small has};
\node[above=0.5cm of note] {\small has};

\end{tikzpicture}
\caption{ER-диаграмма базы данных}
\label{fig:er-diagram-tikz}
\end{figure}

\subsection{Логическая модель}

\begin{figure}[H]\scalebox{0.7}{
\centering
\begin{tikzpicture}[
    node distance=1.5cm,
    auto,
    class/.style={rectangle, draw=green!50, fill=green!10, thick, minimum width=4cm, minimum height=2cm, align=left},
    association/.style={->, >=stealth', shorten >=1pt, thick, dashed}
]

% Классы
\node[class] (contact) {
\textbf{Contact}\\
+ contact\_id: Integer (PK)\\
+ family\_name: String\\
+ work\_place: String\\
+ phone: String\\
+ getMeetings(): List<Meeting>\\
+ getNotes(): List<Note>
};

\node[class, below left=of contact] (meeting) {
\textbf{Meeting}\\
+ meeting\_id: Integer (PK)\\
+ contact\_id: Integer (FK)\\
+ meeting\_time: DateTime\\
+ topic: String\\
+ place: String\\
+ getContact(): Contact
};

\node[class, below right=of contact] (note) {
\textbf{Note}\\
+ note\_id: Integer (PK)\\
+ contact\_id: Integer (FK)\\
+ note\_text: String\\
+ created\_at: DateTime\\
+ getContact(): Contact
};

% Ассоциации
\draw[association] (meeting) -- node[pos=0.5, above, sloped] {1} (contact);
\draw[association] (note) -- node[pos=0.5, above, sloped] {1} (contact);

% Мультипликаторы
\node[above left=0.2cm and 0.2cm of meeting] {\small *};
\node[above right=0.2cm and 0.2cm of note] {\small *};

% Подписи отношений
\node[above=0.3cm of meeting] {\small имеет};
\node[above=0.3cm of note] {\small содержит};

\end{tikzpicture}}
\caption{Логическая модель в виде диаграммы классов UML}
\label{fig:logical-model-tikz}
\end{figure}

\subsection{Физическая модель}

\begin{figure}[H]
\centering\scalebox{0.5}{
\begin{tikzpicture}[
    node distance=1.5cm,
    auto,
    table/.style={rectangle, draw=orange!50, fill=orange!10, thick, minimum width=6cm, minimum height=1.5cm, align=left},
    fk/.style={->, >=stealth', shorten >=1pt, thick, red},
    pk/.style={font=\bfseries\color{blue}},
    fkstyle/.style={font=\itshape\color{red}}
]

% Таблицы
\node[table] (contact) {
\textbf{contact}\\
\pk{contact\_id} SERIAL PRIMARY KEY\\
family\_name VARCHAR(100) NOT NULL\\
work\_place VARCHAR(100)\\
phone VARCHAR(20)
};

\node[table, below=of contact] (meeting) {
\textbf{meeting}\\
\pk{meeting\_id} SERIAL PRIMARY KEY\\
\fkstyle{contact\_id} INTEGER NOT NULL FK\\
meeting\_time TIMESTAMP NOT NULL\\
topic TEXT\\
place TEXT
};

\node[table, below=of meeting] (note) {
\textbf{note}\\
\pk{note\_id} SERIAL PRIMARY KEY\\
\fkstyle{contact\_id} INTEGER NOT NULL FK\\
note\_text TEXT NOT NULL\\
created\_at TIMESTAMP DEFAULT CURRENT\_TIMESTAMP
};

% Внешние ключи
\draw[fk] (meeting.west) to[out=180,in=180] node[pos=0.5, left] {FOREIGN KEY (contact\_id) REFERENCES contact(contact\_id)} (contact.west);
\draw[fk] (note.west) to[out=180,in=180] node[pos=0.5, left] {FOREIGN KEY (contact\_id) REFERENCES contact(contact\_id)} (contact.west);

% Индексы
\node[below right=0.5cm and 0.5cm of meeting] {\small \textbf{Индексы:}};
\node[below right=1cm and 0.5cm of meeting] {\small idx\_meeting\_time ON Meeting(meeting\_time)};
\node[below right=1.5cm and 0.5cm of meeting] {\small idx\_note\_contact\_created ON Note(contact\_id, created\_at)};

\end{tikzpicture}}
\caption{Физическая модель базы данных}
\label{fig:physical-model-tikz}
\end{figure}

\subsection{Нормализация базы данных}

Проведена проверка соответствия нормальным формам:

\subsubsection{Первая нормальная форма (1NF)}
Все атрибуты содержат атомарные значения, повторяющиеся группы отсутствуют.

\subsubsection{Вторая нормальная форма (2NF)}
Все неключевые атрибуты полностью зависят от целого первичного ключа.

\subsubsection{Третья нормальная форма (3NF)}
Отсутствуют транзитивные зависимости неключевых атрибутов от неключевых.

\subsubsection{Нормальная форма Бойса-Кодда (BCNF)}
Каждый детерминант является потенциальным ключом.

\subsection{Реализация бизнес-требований}

Разработаны SQL-запросы для формирования требуемых выходных документов:

\subsubsection{Список предстоящих встреч на неделю}

\begin{minted}[bgcolor=codebg,fontsize=\small]{sql}
SELECT
    c.family_name,
    m.meeting_time,
    m.topic,
    m.place
FROM Meeting m
JOIN Contact c ON m.contact_id = c.contact_id
WHERE m.meeting_time >= CURRENT_DATE
  AND m.meeting_time < CURRENT_DATE + INTERVAL '7 days'
ORDER BY m.meeting_time;
\end{minted}

\subsubsection{История встреч и заметок для контакта}

\begin{minted}[bgcolor=codebg,fontsize=\small]{sql}
SELECT
    m.meeting_time AS event_date,
    'Meeting' AS event_type,
    'Тема: ' || m.topic AS event_description,
    m.place AS details
FROM Meeting m
WHERE m.contact_id = 1

UNION ALL

SELECT
    n.created_at AS event_date,
    'Note' AS event_type,
    n.note_text AS event_description,
    NULL AS details
FROM Note n
WHERE n.contact_id = 1

ORDER BY event_date DESC;
\end{minted}

\section{ЗАКЛЮЧЕНИЕ}

В ходе выполнения лабораторной работы была успешно разработана информационная модель базы данных для системы учета контактов и взаимодействий. Основные достижения и выводы:

\begin{enumerate}
    \item \textbf{Устранены критические ошибки проектирования}: заменены естественные ключи на суррогатные, что обеспечило стабильность структуры данных при изменениях бизнес-информации

    \item \textbf{Оптимизирована структура данных}: упрощена модель хранения заметок, что повысило гибкость системы и упростило запросы

    \item \textbf{Обеспечено соответствие нормальным формам}: проведена полная нормализация базы данных до BCNF, что гарантирует отсутствие аномалий при операциях обновления

    \item \textbf{Создана комплексная документация}: разработаны трехуровневые диаграммы (ER-модель, логическая модель, физическая модель), обеспечивающие наглядное представление структуры базы данных

    \item \textbf{Реализованы бизнес-требования}: разработаны эффективные SQL-запросы для формирования требуемых выходных документов
\end{enumerate}

Полученная модель демонстрирует соответствие лучшим практикам реляционного проектирования и обеспечивает надежную основу для реализации системы личной CRM. Рекомендуется использовать разработанную модель в качестве основы для дальнейшей реализации приложения.

\section{СПИСОК ИСПОЛЬЗОВАННЫХ ИСТОЧНИКОВ}

\begin{enumerate}
    \item Дейт К. Дж. Введение в системы баз данных. — 8-е изд. — М.: Вильямс, 2005. — 1328 с.
    \item PostgreSQL 16.2 Documentation [Электронный ресурс]. — Режим доступа: https://www.postgresql.org/docs/16/index.html (дата обращения: 10.11.2024).
\end{enumerate}

\end{document}

\begin{figure}[H]
\centering
\begin{tikzpicture}[
    node distance=1cm,
    auto,
    box/.style={rectangle, draw=black!50, thick, minimum width=3cm, minimum height=0.8cm, align=center},
    title/.style={font=\bfseries, align=center},
    connector/.style={->, >=stealth', shorten >=1pt, thick, blue}
]

% Заголовок
\node[title] (mainTitle) at (0,0) {Таблицы базы данных};

% CONTACT таблица
\node[title] (contactTitle) at (-4,-1) {Таблица: contact};
\node[box] (contact1) at (-4,-2) {contact\_id: SERIAL PRIMARY KEY};
\node[box] (contact2) at (-4,-3) {family\_name: VARCHAR NOT NULL};
\node[box] (contact3) at (-4,-4) {work\_place: VARCHAR};
\node[box] (contact4) at (-4,-5) {phone: VARCHAR};

% MEETING таблица
\node[title] (meetingTitle) at (0,-1) {Таблица: meeting};
\node[box] (meeting1) at (0,-2) {meeting\_id: SERIAL PRIMARY KEY};
\node[box] (meeting2) at (0,-3) {contact\_id: INTEGER NOT NULL FK};
\node[box] (meeting3) at (0,-4) {meeting\_time: TIMESTAMP NOT NULL};
\node[box] (meeting4) at (0,-5) {topic: TEXT};
\node[box] (meeting5) at (0,-6) {place: TEXT};

% NOTE таблица
\node[title] (noteTitle) at (4,-1) {Таблица: note};
\node[box] (note1) at (4,-2) {note\_id: SERIAL PRIMARY KEY};
\node[box] (note2) at (4,-3) {contact\_id: INTEGER NOT NULL FK};
\node[box] (note3) at (4,-4) {note\_text: TEXT NOT NULL};
\node[box] (note4) at (4,-5) {created\_at: TIMESTAMP DEFAULT NOW};

% Связи
\draw[connector] (meeting2) -- ++(0,-0.5) -| (contact1);
\draw[connector] (note2) -- ++(0,-0.5) -| (contact1);

% Подписи связей
\node[title] at (-2,-7) {Связи 1:N};
\node[title] at (2,-7) {Индексы};
\node[align=left] at (2,-7.5) {idx\_meeting\_time};
\node[align=left] at (2,-8) {idx\_note\_contact\_created};

% Группировка
\draw[rounded corners, dashed] (-5.5,0.5) rectangle (5.5,-8.5);

\end{tikzpicture}
\caption{Детализированная схема базы данных}
\label{fig:detailed-schema-tikz}
\end{figure}



%%% Local Variables:
%%% mode: LaTeX
%%% TeX-master: t
%%% End:
